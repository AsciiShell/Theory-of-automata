%!TEX TS-program = xelatex



\documentclass[a4paper,14pt]{article}

\input{data/preambular.tex}
\begin{document} % конец преамбулы, начало документа
\begin{titlepage}
	\begin{center}
		ФЕДЕРАЛЬНОЕ  ГОСУДАРСТВЕННОЕ АВТОНОМНОЕ \\
		ОБРАЗОВАТЕЛЬНОЕ УЧРЕЖДЕНИЕ ВЫСШЕГО ОБРАЗОВАНИЯ\\
		«НАЦИОНАЛЬНЫЙ ИССЛЕДОВАТЕЛЬСКИЙ УНИВЕРСИТЕТ\\
		«ВЫСШАЯ ШКОЛА ЭКОНОМИКИ»
	\end{center}
	
	\begin{center}
		\textbf{Московский институт электроники и математики}
		
		\textbf{Им. А.Н.Тихонова НИУ ВШЭ}
		
		\vspace{2ex}
		
		\textbf{Департамент компьютерной инженерии}
	\end{center}
	\vspace{1ex}	
	\begin{center}
		Подчезерцев Алексей Евгеньевич, группа БИВ172
		
	\end{center}	
	\vspace{1ex}
	\begin{center}
		\textbf{<<Проектирование многоразрядного десятичного сумматора
			комбинационного типа>>}
	\end{center}	
	\vspace{2ex}
	\begin{center}
		по дисциплине <<Теория автоматов>>

	\end{center}
	\vspace{2ex}
	\begin{flushright}
		Исполнитель:
		
		Студент группы БИВ172
		
		$\rule{5cm}{0.15mm}$ А.Е. Подчезерцев 
		
	\end{flushright}
	\vspace{3ex}
	\begin{flushright}
		Руководитель:
		
		$\rule{5cm}{0.15mm}$ И.И. Бирюков
	\end{flushright}
	\vfill
	\begin{center}
		Москва \the\year \, г.
	\end{center}
\end{titlepage}

\section{Исходные данные для проектирования}


\begin{enumerate}
	\item Количество десятичных разрядов: $3$;
	\item Двоично-десятичный код, в котором находятся числа: $8421+6$;
	\item Система логических элементов: ИЛИ-НЕ, И-НЕ;
	\item Критерий оптимальности элементов для проектирования логических схем: минимальное число логических элементов (ЛЭ) в проектируемых схемах;
	\item Тип триггера для проектирования схемы управления синхронный D-треггер;
	\item Временные параметры синхронизирующей серии импульсов логических элементов: 
	время задержки в любом ЛЭ: 1 нс; 
	импульсы синхронизации длительностью 2 нс со скважностью 1. 
\end{enumerate}

\section{Разработка алгоритма выполнения арифметических операций сложения и вычитания многоразрядных чисел в заданом двоично-десятичном коде}

\begin{table}[H]
	\begin{tabular}{|c|c|}
		\hline
		\multicolumn{1}{|l|}{Цифра} & \multicolumn{1}{l|}{код (8421+6)} \\ \hline
		0 & 0110 \\ \hline
		1 & 0111 \\ \hline
		2 & 1000 \\ \hline
		3 & 1001 \\ \hline
		4 & 1010 \\ \hline
		5 & 1011 \\ \hline
		6 & 1100 \\ \hline
		7 & 1101 \\ \hline
		8 & 1110 \\ \hline
		9 & 1111 \\ \hline
	\end{tabular}
\end{table}

\subsection{Разработка алгоритма для одноразрядных десятичных чисел, получение величины коррекции и критерии ее ввода}

\begin{table}[H]
	\centering
	\begin{tabular}{|c|c|c|c|c|c|c|c|c|c|c|}
		\hline
		     & 0110 & 0111 & 1000 & 1001 & 1010 & 1011 & 1100 & 1101 & 1110 & 1111 \\ \hline
		     & 0110 & 0111 & 1000 & 1001 & 1010 & 1011 & 1100 & 1101 & 1110 & 1111 \\
		0110 & 1100 & 1101 & 1110 & 1111 & 0000 & 0001 & 0010 & 0011 & 0100 & 0101 \\
		     & 1010 & 1010 & 1010 & 1010 & 1010 & 1010 & 1010 & 1010 & 1010 & 1010 \\ \hline
		     & 0111 & 1000 & 1001 & 1010 & 1011 & 1100 & 1101 & 1110 & 1111 & 0110 \\
		0111 & 1101 & 1110 & 1111 & 0000 & 0001 & 0010 & 0011 & 0100 & 0101 & 0110 \\
		     & 1010 & 1010 & 1010 & 1010 & 1010 & 1010 & 1010 & 1010 & 1010 &  -   \\ \hline
		     & 1000 & 1001 & 1010 & 1011 & 1100 & 1101 & 1110 & 1111 & 0110 & 0111 \\
		1000 & 1110 & 1111 & 0000 & 0001 & 0010 & 0011 & 0100 & 0101 & 0110 & 0111 \\
		     & 1010 & 1010 & 1010 & 1010 & 1010 & 1010 & 1010 & 1010 &  -   &  -   \\ \hline
		     & 1001 & 1010 & 1011 & 1100 & 1101 & 1110 & 1111 & 0110 & 0111 & 1000 \\
		1001 & 1111 & 0000 & 0001 & 0010 & 0011 & 0100 & 0101 & 0110 & 0111 & 1000 \\
		     & 1010 & 1010 & 1010 & 1010 & 1010 & 1010 & 1010 &  -   &  -   &  -   \\ \hline
		     & 1010 & 1011 & 1100 & 1101 & 1110 & 1111 & 0110 & 0111 & 1000 & 1001 \\
		1010 & 0000 & 0001 & 0010 & 0011 & 0100 & 0101 & 0110 & 0111 & 1000 & 1001 \\
		     & 1010 & 1010 & 1010 & 1010 & 1010 & 1010 &  -   &  -   &  -   &  -   \\ \hline
		     & 1011 & 1100 & 1101 & 1110 & 1111 & 0110 & 0111 & 1000 & 1001 & 1010 \\
		1011 & 0001 & 0010 & 0011 & 0100 & 0101 & 0110 & 0111 & 1000 & 1001 & 1010 \\
		     & 1010 & 1010 & 1010 & 1010 & 1010 &      &  -   &  -   &  -   &  -   \\ \hline
		     & 1100 & 1101 & 1110 & 1111 & 0110 & 0111 & 1000 & 1001 & 1010 & 1011 \\
		1100 & 0010 & 0011 & 0100 & 0101 & 0110 & 0111 & 1000 & 1001 & 1010 & 1011 \\
		     & 1010 & 1010 & 1010 & 1010 &  -   &  -   &  -   &  -   &  -   &  -   \\ \hline
		     & 1101 & 1110 & 1111 & 0110 & 0111 & 1000 & 1001 & 1010 & 1011 & 1100 \\
		1101 & 0011 & 0100 & 0101 & 0110 & 0111 & 1000 & 1001 & 1010 & 1011 & 1100 \\
		     & 1010 & 1010 & 1010 &  -   &  -   &  -   &  -   &  -   &  -   &  -   \\ \hline
		     & 1110 & 1111 & 0110 & 0111 & 1000 & 1001 & 1010 & 1011 & 1100 & 1101 \\
		1110 & 0100 & 0101 & 0110 & 0111 & 1000 & 1001 & 1010 & 1011 & 1100 & 1101 \\
		     & 1010 & 1010 &  -   &  -   &  -   &  -   &  -   &  -   &  -   &  -   \\ \hline
		     & 1111 & 0110 & 0111 & 1000 & 1001 & 1010 & 1011 & 1100 & 1101 & 1110 \\
		1111 & 0101 & 0110 & 0111 & 1000 & 1001 & 1010 & 1011 & 1100 & 1101 & 1110 \\
		     & 1010 &  -   &  -   &  -   &  -   &  -   &  -   &  -   &  -   &  -   \\ \hline
	\end{tabular}
\end{table}
Критерии ввода корректировки:

\begin{itemize}
	\item Если отсутствует единица переноса или получена недопустимая комбинация, корректировка вводится, единица переноса не вводится;
	\item Если присутствует единица переноса и получена допустимая комбинация, то вводится  корректировка $1010$, при этом единица переноса сохраняется;
\end{itemize}

\subsection{Обобщение полученного алгоритма на многоразрядные числа при выполнении операции сложения и вычитания}

\subsection{Приведение шести примеров на следующие случаи сложения}

\subsubsection{(+A)+(+B)=(+C)}

\begin{figure}[H]
	\centering
	\includegraphics[width=0.7\linewidth]{images/ex1}
	\caption{}
	\label{fig:ex1}
\end{figure}



\subsubsection{(+A)+(-B)=(+C)}

\begin{figure}[H]
	\centering
	\includegraphics[width=0.7\linewidth]{images/ex2}
	\caption{}
	\label{fig:ex2}
\end{figure}



\subsubsection{(+A)+(-B)=(-C)}

\begin{figure}[H]
	\centering
	\includegraphics[width=0.7\linewidth]{images/ex3}
	\caption{}
	\label{fig:ex3}
\end{figure}



\subsubsection{(-A)+(-B)=(-C)}

\begin{figure}[H]
	\centering
	\includegraphics[width=0.7\linewidth]{images/ex4}
	\caption{}
	\label{fig:ex4}
\end{figure}



\subsubsection{(+A)+(+B)=(-C) — Переполнение разрядной сетки}

\begin{figure}[H]
	\centering
	\includegraphics[width=0.7\linewidth]{images/ex5}
	\caption{}
	\label{fig:ex5}
\end{figure}




\subsubsection{(-A)+(-B)=(+C) — Переполнение разрядной сетки}

\begin{figure}[H]
	\centering
	\includegraphics[width=0.7\linewidth]{images/ex6}
	\caption{}
	\label{fig:ex6}
\end{figure}

\section{Разработки функциональной схемы одноразрядного десятичного сумматора комбинационного типа}

\subsection{Разработка оптимальной схемы (с точки зрения критерия оптимальности) одноразрядного двоичного сумматора с учетом заданного базиса логических элементов}

\begin{table}[H]
	\begin{center}
		\caption{\label{tab:tab1} Таблица}
		\begin{tabular}{|c c c|c c|}
			\hline
			$a$ & $b$ & $c$ & $S$ & $P$ \\ \hline
			0 & 0 & 0 & 0 & 0 \\ 
			0 & 0 & 1 & 1 & 0 \\ 
			0 & 1 & 0 & 1 & 0 \\ 
			0 & 1 & 1 & 0 & 1 \\ \hline
			1 & 0 & 0 & 1 & 0 \\ 
			1 & 0 & 1 & 0 & 1 \\ 
			1 & 1 & 0 & 0 & 1 \\ 
			1 & 1 & 1 & 1 & 1 \\ \hline
		\end{tabular}
	\end{center}
\end{table}

\begin{table}[H]
	\begin{center}
		\caption{\label{tab:tab2} Диаграмма $S$}
		\begin{tabular}{ccccc}
			& \multicolumn{2}{c}{$a$}                           & \multicolumn{2}{c}{$\overline{a}$}                          \\ \cline{2-5} 
			\multicolumn{1}{c|}{$b$}  & \multicolumn{1}{c|}{}  & \multicolumn{1}{c|}{1} & \multicolumn{1}{c|}{}  & \multicolumn{1}{c|}{1} \\ \cline{2-5} 
			\multicolumn{1}{c|}{$\overline{b}$} & \multicolumn{1}{c|}{1} & \multicolumn{1}{c|}{}  & \multicolumn{1}{c|}{1} & \multicolumn{1}{c|}{}  \\ \cline{2-5} 
			& $\overline{c}$                     & \multicolumn{2}{c}{$c$}                          & $\overline{c}$                     
		\end{tabular}
	\end{center}
\end{table}

\begin{table}[H]
	\begin{center}
		\caption{\label{tab:tab3} Диаграмма $P$}
		\begin{tabular}{ccccc}
			& \multicolumn{2}{c}{$a$}                           & \multicolumn{2}{c}{$\overline{a}$}                          \\ \cline{2-5} 
			\multicolumn{1}{c|}{$b$}  & \multicolumn{1}{c|}{1}  & \multicolumn{1}{c|}{1} & \multicolumn{1}{c|}{1}  & \multicolumn{1}{c|}{} \\ \cline{2-5} 
			\multicolumn{1}{c|}{$\overline{b}$} & \multicolumn{1}{c|}{} & \multicolumn{1}{c|}{1}  & \multicolumn{1}{c|}{} & \multicolumn{1}{c|}{}  \\ \cline{2-5} 
			& $\overline{c}$                     & \multicolumn{2}{c}{$c$}                          & $\overline{c}$                     
		\end{tabular}
	\end{center}
\end{table}

\begin{table}[H]
	\begin{center}
		\caption{\label{tab:tab4} Диаграмма $\overline{S}$}
		\begin{tabular}{ccccc}
			& \multicolumn{2}{c}{$a$}                           & \multicolumn{2}{c}{$\overline{a}$}                          \\ \cline{2-5} 
			\multicolumn{1}{c|}{$b$}  & \multicolumn{1}{c|}{1}  & \multicolumn{1}{c|}{} & \multicolumn{1}{c|}{1}  & \multicolumn{1}{c|}{} \\ \cline{2-5} 
			\multicolumn{1}{c|}{$\overline{b}$} & \multicolumn{1}{c|}{} & \multicolumn{1}{c|}{1}  & \multicolumn{1}{c|}{} & \multicolumn{1}{c|}{1}  \\ \cline{2-5} 
			& $\overline{c}$                     & \multicolumn{2}{c}{$c$}                          & $\overline{c}$                     
		\end{tabular}
	\end{center}
\end{table}

\begin{table}[H]
	\begin{center}
		\caption{\label{tab:tab5} Диаграмма $\overline{P}$}
		\begin{tabular}{ccccc}
			& \multicolumn{2}{c}{$a$}                           & \multicolumn{2}{c}{$\overline{a}$}                          \\ \cline{2-5} 
			\multicolumn{1}{c|}{$b$}  & \multicolumn{1}{c|}{}  & \multicolumn{1}{c|}{} & \multicolumn{1}{c|}{}  & \multicolumn{1}{c|}{1} \\ \cline{2-5} 
			\multicolumn{1}{c|}{$\overline{b}$} & \multicolumn{1}{c|}{1} & \multicolumn{1}{c|}{}  & \multicolumn{1}{c|}{1} & \multicolumn{1}{c|}{1}  \\ \cline{2-5} 
			& $\overline{c}$                     & \multicolumn{2}{c}{$c$}                          & $\overline{c}$                     
		\end{tabular}
	\end{center}
\end{table}

\begin{equation*}
\begin{aligned}
	S &= abc + a\bar{b}\bar{c} + \bar{a}\bar{b}c + \bar{a}b\bar{c} = \overline{\overline{abc} * \overline{a\bar{b}\bar{c}} * \overline{\bar{a}\bar{b}c} * \overline{\bar{a}b\bar{c}}} - \text{8ЛЭ} \\
	S &= \overline{\overline{(\bar{a}+\bar{b}+c)} + \overline{(\bar{a}+b+\bar{c})} + \overline{(a+\bar{b}+\bar{c})} + \overline{(a+b+c)}} - \text{8ЛЭ}
\end{aligned}
\end{equation*}

\begin{equation*}
\begin{aligned}
	P &= ab + ac + bc = \overline{\overline{ab} * \overline{ac} * \overline{bc}} - \text{4ЛЭ} \\
	P &= \overline{\overline{(a+b)} + \overline{(a+c)} + \overline{(b+c)}} - \text{4ЛЭ}
\end{aligned}
\end{equation*}

\begin{equation*}
	S = \overline{(\bar{P} + abc) (a + b + c)} = \overline{\overline{(\bar{P} + \overline{\overline{abc}})} + \overline{(a+b+c)}} - \text{6ЛЭ}
\end{equation*}


\subsection{Разработка схемы коррекции}

\begin{equation*}
\begin{aligned}
K &= \bar{\text{П'}} + \bar{\gamma_2}\bar{\gamma_8} + \bar{\gamma_8}\bar{\gamma_4} = \overline{\text{П'}*\overline{\bar{\gamma_2}\bar{\gamma_8}}*\overline{\bar{\gamma_8}\bar{\gamma_4}}}- \text{6ЛЭ} \\
K &= \overline{\overline{(\bar{\gamma_8}+\bar{\text{П'}})} + \overline{(\bar{\gamma_4}+\bar{\gamma_2}+\bar{\text{П'}})}} - \text{7ЛЭ}
\end{aligned}
\end{equation*}

\begin{equation*}
\begin{aligned}
\text{П} &= \gamma_8\text{П'} + \gamma_4\gamma_2\text{П'} = \overline{\overline{\gamma_8\text{П'}}*\overline{\gamma_4\gamma_2\text{П'}}} - \text{3ЛЭ} \\
\text{П} &= \overline{\bar{\text{П'}} + \overline{(\gamma_2+\gamma_8)} + \overline{(\gamma_8+\gamma_4)}} - \text{4ЛЭ}
\end{aligned}
\end{equation*}

\subsection{Разработка схемы одноразрядного десятичного сумматора}

\section{Разработка дополнительных схем для функционирования многоразрядного десятичного сумматора}

\subsection{Разработка преобразователя прямого кода в обратный для работы с отрицательными величинами}

\begin{equation*}
\begin{aligned}
\widehat{A_8} &= \bar{d} + \bar{c} + \bar{a}b + a\bar{b} = \overline{d*c*\overline{\bar{a}b}*\overline{a\bar{b}}} - \text{5ЛЭ} \\
\widehat{A_8} &= \overline{abc+\bar{a}\bar{b}} = \overline{\overline{(\bar{a}+\bar{b}+\bar{c})}+\overline{(a+b)}} - \text{6ЛЭ}
\end{aligned}
\end{equation*}

\begin{equation*}
\begin{aligned}
\widehat{A_4} &= a\bar{c}\bar{d} + cd + \bar{a}c = \overline{\overline{a\bar{c}\bar{d}}*\overline{cd}*\overline{\bar{a}c}} - \text{7ЛЭ} \\
\widehat{A_4} &= \overline{ac\bar{d} + \bar{a}\bar{c}}  = 
\overline{\overline{(a+\bar{c}+d)+\overline{(a+c)}}}- \text{4ЛЭ}
\end{aligned}
\end{equation*}

\begin{equation*}
\begin{aligned}
\widehat{A_2} &= d - \text{0ЛЭ}
\end{aligned}
\end{equation*}

\begin{equation*}
\begin{aligned}
\widehat{A_1} &= \bar{a}e + a\bar{e} = \overline{\overline{\bar{a}e}*a\bar{e}} - \text{5ЛЭ} \\
\widehat{A_1} &= \overline{ae+\bar{a}\bar{e}} = \overline{\overline{(\bar{a}+\bar{e})}+\overline{(a+e)}} - \text{5ЛЭ}
\end{aligned}
\end{equation*}
\subsection{Разработка схемы, фиксирующей переполнение разрядной сетки}

\subsection{Разработка схемы для определения знака суммы}

\section{Разработки функциональной схемы многоразрядного десятичного сумматора}

\section{Разработка устройства управления для многоразрядного десятичного сумматора}

\subsection{Разработка входных и выходных регистров хранения числовой информации, участвующей в операции сложения}

\subsection{Разработка регистра признаков результата}

\subsection{Расчет временных параметров устройства управления}

\subsection{Разработка схемы для получения управляющих сигналов и схемы пуска выполнения операции сложения}

\section{Общая структура схемы многоразрядного десятичного сумматора комбинационного типа с устройством управления}

\section{Выводы по работе}

\end{document}