%!TEX TS-program = xelatex



\documentclass[a4paper,14pt]{article}

\input{data/preambular.tex}
\begin{document} % конец преамбулы, начало документа
\begin{titlepage}
	\begin{center}
		ФЕДЕРАЛЬНОЕ  ГОСУДАРСТВЕННОЕ АВТОНОМНОЕ \\
		ОБРАЗОВАТЕЛЬНОЕ УЧРЕЖДЕНИЕ ВЫСШЕГО ОБРАЗОВАНИЯ\\
		«НАЦИОНАЛЬНЫЙ ИССЛЕДОВАТЕЛЬСКИЙ УНИВЕРСИТЕТ\\
		«ВЫСШАЯ ШКОЛА ЭКОНОМИКИ»
	\end{center}
	
	\begin{center}
		\textbf{Московский институт электроники и математики}
		
		\textbf{Им. А.Н.Тихонова НИУ ВШЭ}
		
		\vspace{2ex}
		
		\textbf{Департамент компьютерной инженерии}
	\end{center}
	\vspace{1ex}	
	\begin{center}
		Подчезерцев Алексей Евгеньевич, группа БИВ172
		
	\end{center}	
	\vspace{1ex}
	\begin{center}
		\textbf{<<Проектирование многоразрядного десятичного сумматора
			комбинационного типа>>}
	\end{center}	
	\vspace{2ex}
	\begin{center}
		по дисциплине <<Теория автоматов>>

	\end{center}
	\vspace{2ex}
	\begin{flushright}
		Исполнитель:
		
		Студент группы БИВ172
		
		$\rule{5cm}{0.15mm}$ А.Е. Подчезерцев 
		
	\end{flushright}
	\vspace{3ex}
	\begin{flushright}
		Руководитель:
		
		$\rule{5cm}{0.15mm}$ И.И. Бирюков
	\end{flushright}
	\vfill
	\begin{center}
		Москва \the\year \, г.
	\end{center}
\end{titlepage}

\section{Исходные данные для проектирования}


\begin{enumerate}
	\item Количество десятичных разрядов: $3$;
	\item Двоично-десятичный код, в котором находятся числа: $8421+6$;
	\item Система логических элементов: ИЛИ-НЕ, И-НЕ;
	\item Критерий оптимальности элементов для проектирования логических схем: минимальное число логических элементов (ЛЭ) в проектируемых схемах;
	\item Тип триггера для проектирования схемы управления синхронный D-треггер;
	\item Временные параметры синхронизирующей серии импульсов логических элементов: 
	время задержки в любом ЛЭ: 1 нс; 
	импульсы синхронизации длительностью 2 нс со скважностью 1. 
\end{enumerate}

\section{Разработка алгоритма выполнения арифметических операций сложения и вычитания многоразрядных чисел в заданом двоично-десятичном коде}

\begin{table}[H]
	\begin{tabular}{|c|c|}
		\hline
		\multicolumn{1}{|l|}{Цифра} & \multicolumn{1}{l|}{код (8421+6)} \\ \hline
		0 & 0110 \\ \hline
		1 & 0111 \\ \hline
		2 & 1000 \\ \hline
		3 & 1001 \\ \hline
		4 & 1010 \\ \hline
		5 & 1011 \\ \hline
		6 & 1100 \\ \hline
		7 & 1101 \\ \hline
		8 & 1110 \\ \hline
		9 & 1111 \\ \hline
	\end{tabular}
\end{table}

\subsection{Разработка алгоритма для одноразрядных десятичных чисел, получение величины коррекции и критерии ее ввода}

\begin{table}[H]
	\centering
	\begin{tabular}{|c|c|c|c|c|c|c|c|c|c|c|}
		\hline
		     & 0110 & 0111 & 1000 & 1001 & 1010 & 1011 & 1100 & 1101 & 1110 & 1111 \\ \hline
		     & 0110 & 0111 & 1000 & 1001 & 1010 & 1011 & 1100 & 1101 & 1110 & 1111 \\
		0110 & 1100 & 1101 & 1110 & 1111 & 0000 & 0001 & 0010 & 0011 & 0100 & 0101 \\
		     & 1010 & 1010 & 1010 & 1010 & 1010 & 1010 & 1010 & 1010 & 1010 & 1010 \\ \hline
		     & 0111 & 1000 & 1001 & 1010 & 1011 & 1100 & 1101 & 1110 & 1111 & 0110 \\
		0111 & 1101 & 1110 & 1111 & 0000 & 0001 & 0010 & 0011 & 0100 & 0101 & 0110 \\
		     & 1010 & 1010 & 1010 & 1010 & 1010 & 1010 & 1010 & 1010 & 1010 &  -   \\ \hline
		     & 1000 & 1001 & 1010 & 1011 & 1100 & 1101 & 1110 & 1111 & 0110 & 0111 \\
		1000 & 1110 & 1111 & 0000 & 0001 & 0010 & 0011 & 0100 & 0101 & 0110 & 0111 \\
		     & 1010 & 1010 & 1010 & 1010 & 1010 & 1010 & 1010 & 1010 &  -   &  -   \\ \hline
		     & 1001 & 1010 & 1011 & 1100 & 1101 & 1110 & 1111 & 0110 & 0111 & 1000 \\
		1001 & 1111 & 0000 & 0001 & 0010 & 0011 & 0100 & 0101 & 0110 & 0111 & 1000 \\
		     & 1010 & 1010 & 1010 & 1010 & 1010 & 1010 & 1010 &  -   &  -   &  -   \\ \hline
		     & 1010 & 1011 & 1100 & 1101 & 1110 & 1111 & 0110 & 0111 & 1000 & 1001 \\
		1010 & 0000 & 0001 & 0010 & 0011 & 0100 & 0101 & 0110 & 0111 & 1000 & 1001 \\
		     & 1010 & 1010 & 1010 & 1010 & 1010 & 1010 &  -   &  -   &  -   &  -   \\ \hline
		     & 1011 & 1100 & 1101 & 1110 & 1111 & 0110 & 0111 & 1000 & 1001 & 1010 \\
		1011 & 0001 & 0010 & 0011 & 0100 & 0101 & 0110 & 0111 & 1000 & 1001 & 1010 \\
		     & 1010 & 1010 & 1010 & 1010 & 1010 &      &  -   &  -   &  -   &  -   \\ \hline
		     & 1100 & 1101 & 1110 & 1111 & 0110 & 0111 & 1000 & 1001 & 1010 & 1011 \\
		1100 & 0010 & 0011 & 0100 & 0101 & 0110 & 0111 & 1000 & 1001 & 1010 & 1011 \\
		     & 1010 & 1010 & 1010 & 1010 &  -   &  -   &  -   &  -   &  -   &  -   \\ \hline
		     & 1101 & 1110 & 1111 & 0110 & 0111 & 1000 & 1001 & 1010 & 1011 & 1100 \\
		1101 & 0011 & 0100 & 0101 & 0110 & 0111 & 1000 & 1001 & 1010 & 1011 & 1100 \\
		     & 1010 & 1010 & 1010 &  -   &  -   &  -   &  -   &  -   &  -   &  -   \\ \hline
		     & 1110 & 1111 & 0110 & 0111 & 1000 & 1001 & 1010 & 1011 & 1100 & 1101 \\
		1110 & 0100 & 0101 & 0110 & 0111 & 1000 & 1001 & 1010 & 1011 & 1100 & 1101 \\
		     & 1010 & 1010 &  -   &  -   &  -   &  -   &  -   &  -   &  -   &  -   \\ \hline
		     & 1111 & 0110 & 0111 & 1000 & 1001 & 1010 & 1011 & 1100 & 1101 & 1110 \\
		1111 & 0101 & 0110 & 0111 & 1000 & 1001 & 1010 & 1011 & 1100 & 1101 & 1110 \\
		     & 1010 &  -   &  -   &  -   &  -   &  -   &  -   &  -   &  -   &  -   \\ \hline
	\end{tabular}
\end{table}
Критерии ввода корректировки:

\begin{itemize}
	\item Если получена разрешенная комбинация и вне зависимости от наличия единицы переноса, корректировка не вводится, при этом единица переноса сохраняется;
	\item Если получена запрещенная комбинация и нет единицы переноса, то вводится корректировка $0110$;
	\item Если получена запрещенная комбинация и есть единица переноса, то вводится корректировка $1010$, при этом единица переноса сохраняется;
\end{itemize}

\subsection{Обобщение полученного алгоритма на многоразрядные числа при выполнении операции сложения и вычитания}

\subsection{Приведение шести примеров на следующие случаи сложения}

\subsubsection{(+A)+(+B)=(+C)}

%\begin{figure}[H]
%	\centering
%	\includegraphics[width=0.2\linewidth]{images/2_3_1_01}
%	\caption{}
%	\label{fig:23101}
%\end{figure}



\subsubsection{(+A)+(-B)=(+C)}

%\begin{figure}[H]
%	\centering
%	\includegraphics[width=0.4\linewidth]{images/2_3_2_01}
%	\caption{}
%	\label{fig:23201}
%\end{figure}



\subsubsection{(+A)+(-B)=(-C)}

%\begin{figure}[H]
%	\centering
%	\includegraphics[width=0.4\linewidth]{images/2_3_3_01}
%	\caption{}
%	\label{fig:23301}
%\end{figure}



\subsubsection{(-A)+(-B)=(-C)}

%\begin{figure}[H]
%	\centering
%	\includegraphics[width=0.4\linewidth]{images/2_3_4_01}
%	\caption{}
%	\label{fig:23401}
%\end{figure}



\subsubsection{(+A)+(+B)=(-C) — Переполнение разрядной сетки}

%\begin{figure}[H]
%	\centering
%	\includegraphics[width=0.4\linewidth]{images/2_3_5_01}
%	\caption{}
%	\label{fig:23501}
%\end{figure}




\subsubsection{(-A)+(-B)=(+C) — Переполнение разрядной сетки}

%\begin{figure}[H]
%	\centering
%	\includegraphics[width=0.4\linewidth]{images/2_3_6_01}
%	\caption{}
%	\label{fig:23601}
%\end{figure}

\section{Разработки функциональной схемы одноразрядного десятичного сумматора комбинационного типа}

\subsection{Разработка оптимальной схемы (с точки зрения критерия оптимальности) одноразрядного двоичного сумматора с учетом заданного базиса логических элементов}

\subsection{Разработка схемы коррекции}

\subsection{Разработка схемы одноразрядного десятичного сумматора}

\section{Разработка дополнительных схем для функционирования многоразрядного десятичного сумматора}

\subsection{Разработка преобразователя прямого кода в обратный для работы с отрицательными величинами}

\subsection{Разработка схемы, фиксирующей переполнение разрядной сетки}

\subsection{Разработка схемы для определения знака суммы}

\section{Разработки функциональной схемы многоразрядного десятичного сумматора}

\section{Разработка устройства управления для многоразрядного десятичного сумматора}

\subsection{Разработка входных и выходных регистров хранения числовой информации, участвующей в операции сложения}

\subsection{Разработка регистра признаков результата}

\subsection{Расчет временных параметров устройства управления}

\subsection{Разработка схемы для получения управляющих сигналов и схемы пуска выполнения операции сложения}

\section{Общая структура схемы многоразрядного десятичного сумматора комбинационного типа с устройством управления}

\section{Выводы по работе}

\end{document}