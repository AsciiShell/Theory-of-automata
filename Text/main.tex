%!TEX TS-program = xelatex



\documentclass[a4paper,14pt]{article}


%%% Работа с русским языком
\usepackage[english,russian]{babel}   %% загружает пакет многоязыковой вёрстки
\usepackage{fontspec}      %% подготавливает загрузку шрифтов Open Type, True Type и др.
\defaultfontfeatures{Ligatures={TeX},Renderer=Basic}  %% свойства шрифтов по умолчанию
\setmainfont[Ligatures={TeX,Historic}]{Times New Roman} %% задаёт основной шрифт документа
\setsansfont{Comic Sans MS}                    %% задаёт шрифт без засечек
\setmonofont{Courier New}
\usepackage{indentfirst}
\frenchspacing

\renewcommand{\epsilon}{\ensuremath{\varepsilon}}
\renewcommand{\phi}{\ensuremath{\varphi}}
\renewcommand{\kappa}{\ensuremath{\varkappa}}
\renewcommand{\le}{\ensuremath{\leqslant}}
\renewcommand{\leq}{\ensuremath{\leqslant}}
\renewcommand{\ge}{\ensuremath{\geqslant}}
\renewcommand{\geq}{\ensuremath{\geqslant}}
\renewcommand{\emptyset}{\varnothing}

%%% Дополнительная работа с математикой
\usepackage{amsmath,amsfonts,amssymb,amsthm,mathtools} % AMS
\usepackage{icomma} % "Умная" запятая: $0,2$ --- число, $0, 2$ --- перечисление

%% Номера формул
%\mathtoolsset{showonlyrefs=true} % Показывать номера только у тех формул, на которые есть \eqref{} в тексте.
%\usepackage{leqno} % Нумерация формул слева	

%% Перенос знаков в формулах (по Львовскому)
\newcommand*{\hm}[1]{#1\nobreak\discretionary{}
	{\hbox{$\mathsurround=0pt #1$}}{}}

%%% Работа с картинками
\usepackage{graphicx}  % Для вставки рисунков
\graphicspath{{images/}}  % папки с картинками
\setlength\fboxsep{3pt} % Отступ рамки \fbox{} от рисунка
\setlength\fboxrule{1pt} % Толщина линий рамки \fbox{}
\usepackage{wrapfig} % Обтекание рисунков текстом

%%% Работа с таблицами
\usepackage{array,tabularx,tabulary,booktabs} % Дополнительная работа с таблицами
\usepackage{longtable}  % Длинные таблицы
\usepackage{multirow} % Слияние строк в таблице
\usepackage{float}% http://ctan.org/pkg/float

%%% Программирование
\usepackage{etoolbox} % логические операторы


%%% Страница
\usepackage{extsizes} % Возможность сделать 14-й шрифт
\usepackage{geometry} % Простой способ задавать поля
\geometry{top=20mm}
\geometry{bottom=20mm}
\geometry{left=20mm}
\geometry{right=10mm}
%
%\usepackage{fancyhdr} % Колонтитулы
% 	\pagestyle{fancy}
%\renewcommand{\headrulewidth}{0pt}  % Толщина линейки, отчеркивающей верхний колонтитул
% 	\lfoot{Нижний левый}
% 	\rfoot{Нижний правый}
% 	\rhead{Верхний правый}
% 	\chead{Верхний в центре}
% 	\lhead{Верхний левый}
%	\cfoot{Нижний в центре} % По умолчанию здесь номер страницы

\usepackage{setspace} % Интерлиньяж
\onehalfspacing % Интерлиньяж 1.5
%\doublespacing % Интерлиньяж 2
%\singlespacing % Интерлиньяж 1

\usepackage{lastpage} % Узнать, сколько всего страниц в документе.

\usepackage{soul} % Модификаторы начертания

\usepackage{hyperref}
\usepackage[usenames,dvipsnames,svgnames,table,rgb]{xcolor}
\hypersetup{				% Гиперссылки
	unicode=true,           % русские буквы в раздела PDF
	pdftitle={Проектирование многоразрядного десятичного сумматора комбинационного типа},   % Заголовок
	pdfauthor={Подчезерцев Алексей},      % Автор
	pdfsubject={Проектирование многоразрядного десятичного сумматора комбинационного типа},      % Тема
	pdfcreator={Подчезерцев Алексей}, % Создатель
	pdfproducer={Подчезерцев Алексей}, % Производитель
	pdfkeywords={Теория автоматов} {МИЭМ} {ВШЭ}, % Ключевые слова
	colorlinks=true,       	% false: ссылки в рамках; true: цветные ссылки
	linkcolor=black,          % внутренние ссылки
	citecolor=black,        % на библиографию
	filecolor=magenta,      % на файлы
	urlcolor=black           % на URL
}
\makeatletter 
\def\@biblabel#1{#1. } 
\makeatother
\usepackage{cite} % Работа с библиографией
%\usepackage[superscript]{cite} % Ссылки в верхних индексах
%\usepackage[nocompress]{cite} % 
\usepackage{csquotes} % Еще инструменты для ссылок

\usepackage{multicol} % Несколько колонок

\usepackage{tikz} % Работа с графикой
\usepackage{pgfplots}
\usepackage{pgfplotstable}

% ГОСТ заголовки
\usepackage[font=small]{caption}
%\captionsetup[table]{justification=centering, labelsep = newline} % Таблицы по правобу краю
%\captionsetup[figure]{justification=centering} % Картинки по центру


\newcommand{\tablecaption}[1]{\addtocounter{table}{1}\small \begin{flushright}\tablename \ \thetable\end{flushright}%	
\begin{center}#1\end{center}}

\newcommand{\imref}[1]{рис.~\ref{#1}}

\usepackage{multirow}
\usepackage{spreadtab}
\newcolumntype{K}[1]{@{}>{\centering\arraybackslash}p{#1cm}@{}}


\usepackage{xparse}
\ExplSyntaxOn
\DeclareExpandableDocumentCommand{\juliandate}{ m m m }
{
	\juliandate_calc:nnnn { #1 } { #2 } { #3 } { \use:n }
}
\NewDocumentCommand{\storejuliandate}{ s m m m m }
{
	\IfBooleanTF{#1}
	{
		\juliandate_calc:nnnn { #3 } { #4 } { #5 } { \cs_set:Npx #2 }
	}
	{
		\juliandate_calc:nnnn { #3 } { #4 } { #5 } { \cs_new:Npx #2 }
	}
}
\cs_new:Npn \juliandate_calc:nnnn #1 #2 #3 #4 % #1 = day, #2 = month, #3 = year, #4 = what to do
{
	#4 
	{
		\int_eval:n
		{
			#1 +
			\int_div_truncate:nn { 153 * (#2 + 12 * \int_div_truncate:nn { 14 - #2 } { 12 } - 3) + 2 } { 5 } +
			365 * (#3 + 4800 - \int_div_truncate:nn { 14 - #2 } { 12 } ) +
			\int_div_truncate:nn { #3 + 4800 - \int_div_truncate:nn { 14 - #2 } { 12 } } { 4 } -
			\int_div_truncate:nn { #3 + 4800 - \int_div_truncate:nn { 14 - #2 } { 12 } } { 100 } + 
			\int_div_truncate:nn { #3 + 4800 - \int_div_truncate:nn { 14 - #2 } { 12 } } { 400 } -
			32045
		}
	}
}

\tl_new:N \l__juliandate_g_tl
\tl_new:N \l__juliandate_dg_tl
\tl_new:N \l__juliandate_c_tl
\tl_new:N \l__juliandate_dc_tl
\tl_new:N \l__juliandate_b_tl
\tl_new:N \l__juliandate_db_tl
\tl_new:N \l__juliandate_a_tl
\tl_new:N \l__juliandate_da_tl
\tl_new:N \l__juliandate_y_tl
\tl_new:N \l__juliandate_m_tl
\tl_new:N \l__juliandate_d_tl
\int_new:N \l_juliandate_day_int
\int_new:N \l_juliandate_month_int
\int_new:N \l_juliandate_year_int

\cs_new:Npn \__juliandate_set:nn #1 #2
{
	\tl_set:cx { l__juliandate_#1_tl } { \int_eval:n { #2 } }
}
\cs_new:Npn \__juliandate_use:n #1
{
	\tl_use:c { l__juliandate_#1_tl }
}
\cs_new_protected:Npn \juliandate_reverse:n #1
{
	\__juliandate_set:nn { g }
	{ \int_div_truncate:nn { #1 + 32044 } { 146097 } }
	\__juliandate_set:nn { dg }
	{ \int_mod:nn { #1 + 32044 } { 146097 } }
	\__juliandate_set:nn { c }
	{ \int_div_truncate:nn { ( \int_div_truncate:nn { \__juliandate_use:n { dg } } { 36524 } + 1) * 3 } { 4 } }
	\__juliandate_set:nn { dc }
	{ \__juliandate_use:n { dg } - \__juliandate_use:n { c } * 36524 }
	\__juliandate_set:nn { b }
	{ \int_div_truncate:nn { \__juliandate_use:n { dc } } { 1461 } }
	\__juliandate_set:nn { db }
	{ \int_mod:nn { \__juliandate_use:n { dc } } { 1461 } }
	\__juliandate_set:nn { a }
	{ \int_div_truncate:nn { ( \int_div_truncate:nn { \__juliandate_use:n { db } } { 365 } + 1) * 3 } { 4 } }
	\__juliandate_set:nn { da }
	{ \__juliandate_use:n { db } - \__juliandate_use:n { a } * 365 }
	\__juliandate_set:nn { y }
	{
		\__juliandate_use:n { g } * 400 + 
		\__juliandate_use:n { c } * 100 + 
		\__juliandate_use:n { b } * 4 + 
		\__juliandate_use:n { a }
	}
	\__juliandate_set:nn { m }
	{ \int_div_truncate:nn { \__juliandate_use:n { da } * 5 + 308 } { 153 } - 2 }
	\__juliandate_set:nn { d }
	{ \__juliandate_use:n { da } - \int_div_truncate:nn { (\__juliandate_use:n { m } + 4) * 153 } { 5 } + 122 }
	\int_set:Nn \l_juliandate_year_int
	{ \__juliandate_use:n { y } - 4800 + \int_div_truncate:nn { \__juliandate_use:n { m } + 2 } { 12 } }
	\int_set:Nn \l_juliandate_month_int
	{ \int_mod:nn { \__juliandate_use:n { m } + 2 } { 12 } + 1 }
	\int_set:Nn \l_juliandate_day_int
	{ \__juliandate_use:n { d } + 1 }
}
\cs_generate_variant:Nn \juliandate_reverse:n { x }

\NewDocumentCommand{\showday}{ m }
{
	\juliandate_reverse:n { #1 }
	\int_to_arabic:n { \l_juliandate_day_int }-
	\int_to_arabic:n { \l_juliandate_month_int }-
	\int_to_arabic:n { \l_juliandate_year_int }
}

\NewDocumentCommand{\tomorrow}{ }
{
	\group_begin:
	\juliandate_reverse:x { \juliandate_calc:nnnn { \day + 1 } { \month } { \year } { \use:n } }
	\day = \l_juliandate_day_int
	\month = \l_juliandate_month_int
	\year = \l_juliandate_year_int
	\today
	\group_end:
}
\NewDocumentCommand{\tomorrowof}{ m m m }
{
	\group_begin:
	\juliandate_reverse:x { \juliandate_calc:nnnn { #1 + 1 } { #2 } { #3 } { \use:n } }
	\day = \l_juliandate_day_int
	\month = \l_juliandate_month_int
	\year = \l_juliandate_year_int
	\today
	\group_end:
}
\ExplSyntaxOff

\RequirePackage{lscape}
\usepackage{pdflscape}




\usepackage[figure,table,page]{totalcount} 
\usepackage{lastpage} 
\makeatletter 
\long\def\@secondoffour#1#2#3#4{#2} 
\def\getlastpage{\ifx\r@LastPage\@undefined 0\else 
	\expandafter\@secondoffour\r@LastPage\@empty\@empty\fi} 
\makeatother 

\begin{document} % конец преамбулы, начало документа
\begin{titlepage}
	\begin{center}
		ФЕДЕРАЛЬНОЕ  ГОСУДАРСТВЕННОЕ АВТОНОМНОЕ \\
		ОБРАЗОВАТЕЛЬНОЕ УЧРЕЖДЕНИЕ ВЫСШЕГО ОБРАЗОВАНИЯ\\
		«НАЦИОНАЛЬНЫЙ ИССЛЕДОВАТЕЛЬСКИЙ УНИВЕРСИТЕТ\\
		«ВЫСШАЯ ШКОЛА ЭКОНОМИКИ»
	\end{center}
	
	\begin{center}
		\textbf{Московский институт электроники и математики}
		
		\textbf{Им. А.Н.Тихонова НИУ ВШЭ}
		
		\vspace{2ex}
		
		\textbf{Департамент компьютерной инженерии}
	\end{center}
	\vspace{1ex}	
	\begin{center}
		Подчезерцев Алексей Евгеньевич, группа БИВ172
		
	\end{center}	
	\vspace{1ex}
	\begin{center}
		\textbf{<<Проектирование многоразрядного десятичного сумматора
			комбинационного типа>>}
	\end{center}	
	\vspace{2ex}
	\begin{center}
		по дисциплине <<Теория автоматов>>

	\end{center}
	\vspace{2ex}
	\begin{flushright}
		Исполнитель:
		
		Студент группы БИВ172
		
		$\rule{5cm}{0.15mm}$ А.Е. Подчезерцев 
		
	\end{flushright}
	\vspace{3ex}
	\begin{flushright}
		Руководитель:
		
		$\rule{5cm}{0.15mm}$ И.И. Бирюков
	\end{flushright}
	\vfill
	\begin{center}
		Москва \the\year \, г.
	\end{center}
\end{titlepage}

\section{Исходные данные для проектирования}


\begin{enumerate}
	\item Количество десятичных разрядов: $3$;
	\item Двоично-десятичный код, в котором находятся числа: $8421+6$;
	\item Система логических элементов: ИЛИ-НЕ, И-НЕ;
	\item Критерий оптимальности элементов для проектирования логических схем: минимальное число логических элементов (ЛЭ) в проектируемых схемах;
	\item Тип триггера для проектирования схемы управления синхронный D-треггер;
	\item Временные параметры синхронизирующей серии импульсов логических элементов: 
	время задержки в любом ЛЭ: 1 нс; 
	импульсы синхронизации длительностью 2 нс со скважностью 1. 
\end{enumerate}

\section{Разработка алгоритма выполнения арифметических операций сложения и вычитания многоразрядных чисел в заданом двоично-десятичном коде}

\begin{table}[H]
	\begin{tabular}{|c|c|}
		\hline
		\multicolumn{1}{|l|}{Цифра} & \multicolumn{1}{l|}{код (8421+6)} \\ \hline
		0 & 0110 \\ \hline
		1 & 0111 \\ \hline
		2 & 1000 \\ \hline
		3 & 1001 \\ \hline
		4 & 1010 \\ \hline
		5 & 1011 \\ \hline
		6 & 1100 \\ \hline
		7 & 1101 \\ \hline
		8 & 1110 \\ \hline
		9 & 1111 \\ \hline
	\end{tabular}
\end{table}

\subsection{Разработка алгоритма для одноразрядных десятичных чисел, получение величины коррекции и критерии ее ввода}

\begin{table}[H]
	\centering
	\begin{tabular}{|c|c|c|c|c|c|c|c|c|c|c|}
		\hline
		     & 0110 & 0111 & 1000 & 1001 & 1010 & 1011 & 1100 & 1101 & 1110 & 1111 \\ \hline
		     & 0110 & 0111 & 1000 & 1001 & 1010 & 1011 & 1100 & 1101 & 1110 & 1111 \\
		0110 & 1100 & 1101 & 1110 & 1111 & 0000 & 0001 & 0010 & 0011 & 0100 & 0101 \\
		     & 1010 & 1010 & 1010 & 1010 & 1010 & 1010 & 1010 & 1010 & 1010 & 1010 \\ \hline
		     & 0111 & 1000 & 1001 & 1010 & 1011 & 1100 & 1101 & 1110 & 1111 & 0110 \\
		0111 & 1101 & 1110 & 1111 & 0000 & 0001 & 0010 & 0011 & 0100 & 0101 & 0110 \\
		     & 1010 & 1010 & 1010 & 1010 & 1010 & 1010 & 1010 & 1010 & 1010 &  -   \\ \hline
		     & 1000 & 1001 & 1010 & 1011 & 1100 & 1101 & 1110 & 1111 & 0110 & 0111 \\
		1000 & 1110 & 1111 & 0000 & 0001 & 0010 & 0011 & 0100 & 0101 & 0110 & 0111 \\
		     & 1010 & 1010 & 1010 & 1010 & 1010 & 1010 & 1010 & 1010 &  -   &  -   \\ \hline
		     & 1001 & 1010 & 1011 & 1100 & 1101 & 1110 & 1111 & 0110 & 0111 & 1000 \\
		1001 & 1111 & 0000 & 0001 & 0010 & 0011 & 0100 & 0101 & 0110 & 0111 & 1000 \\
		     & 1010 & 1010 & 1010 & 1010 & 1010 & 1010 & 1010 &  -   &  -   &  -   \\ \hline
		     & 1010 & 1011 & 1100 & 1101 & 1110 & 1111 & 0110 & 0111 & 1000 & 1001 \\
		1010 & 0000 & 0001 & 0010 & 0011 & 0100 & 0101 & 0110 & 0111 & 1000 & 1001 \\
		     & 1010 & 1010 & 1010 & 1010 & 1010 & 1010 &  -   &  -   &  -   &  -   \\ \hline
		     & 1011 & 1100 & 1101 & 1110 & 1111 & 0110 & 0111 & 1000 & 1001 & 1010 \\
		1011 & 0001 & 0010 & 0011 & 0100 & 0101 & 0110 & 0111 & 1000 & 1001 & 1010 \\
		     & 1010 & 1010 & 1010 & 1010 & 1010 &      &  -   &  -   &  -   &  -   \\ \hline
		     & 1100 & 1101 & 1110 & 1111 & 0110 & 0111 & 1000 & 1001 & 1010 & 1011 \\
		1100 & 0010 & 0011 & 0100 & 0101 & 0110 & 0111 & 1000 & 1001 & 1010 & 1011 \\
		     & 1010 & 1010 & 1010 & 1010 &  -   &  -   &  -   &  -   &  -   &  -   \\ \hline
		     & 1101 & 1110 & 1111 & 0110 & 0111 & 1000 & 1001 & 1010 & 1011 & 1100 \\
		1101 & 0011 & 0100 & 0101 & 0110 & 0111 & 1000 & 1001 & 1010 & 1011 & 1100 \\
		     & 1010 & 1010 & 1010 &  -   &  -   &  -   &  -   &  -   &  -   &  -   \\ \hline
		     & 1110 & 1111 & 0110 & 0111 & 1000 & 1001 & 1010 & 1011 & 1100 & 1101 \\
		1110 & 0100 & 0101 & 0110 & 0111 & 1000 & 1001 & 1010 & 1011 & 1100 & 1101 \\
		     & 1010 & 1010 &  -   &  -   &  -   &  -   &  -   &  -   &  -   &  -   \\ \hline
		     & 1111 & 0110 & 0111 & 1000 & 1001 & 1010 & 1011 & 1100 & 1101 & 1110 \\
		1111 & 0101 & 0110 & 0111 & 1000 & 1001 & 1010 & 1011 & 1100 & 1101 & 1110 \\
		     & 1010 &  -   &  -   &  -   &  -   &  -   &  -   &  -   &  -   &  -   \\ \hline
	\end{tabular}
\end{table}
Критерии ввода корректировки:

\begin{itemize}
	\item Если отсутствует единица переноса или получена недопустимая комбинация, корректировка вводится, единица переноса не вводится;
	\item Если присутствует единица переноса и получена допустимая комбинация, то вводится  корректировка $1010$, при этом единица переноса сохраняется;
\end{itemize}

\subsection{Обобщение полученного алгоритма на многоразрядные числа при выполнении операции сложения и вычитания}

\subsection{Приведение шести примеров на следующие случаи сложения}

\subsubsection{(+A)+(+B)=(+C)}

%\begin{figure}[H]
%	\centering
%	\includegraphics[width=0.2\linewidth]{images/2_3_1_01}
%	\caption{}
%	\label{fig:23101}
%\end{figure}



\subsubsection{(+A)+(-B)=(+C)}

%\begin{figure}[H]
%	\centering
%	\includegraphics[width=0.4\linewidth]{images/2_3_2_01}
%	\caption{}
%	\label{fig:23201}
%\end{figure}



\subsubsection{(+A)+(-B)=(-C)}

%\begin{figure}[H]
%	\centering
%	\includegraphics[width=0.4\linewidth]{images/2_3_3_01}
%	\caption{}
%	\label{fig:23301}
%\end{figure}



\subsubsection{(-A)+(-B)=(-C)}

%\begin{figure}[H]
%	\centering
%	\includegraphics[width=0.4\linewidth]{images/2_3_4_01}
%	\caption{}
%	\label{fig:23401}
%\end{figure}



\subsubsection{(+A)+(+B)=(-C) — Переполнение разрядной сетки}

%\begin{figure}[H]
%	\centering
%	\includegraphics[width=0.4\linewidth]{images/2_3_5_01}
%	\caption{}
%	\label{fig:23501}
%\end{figure}




\subsubsection{(-A)+(-B)=(+C) — Переполнение разрядной сетки}

%\begin{figure}[H]
%	\centering
%	\includegraphics[width=0.4\linewidth]{images/2_3_6_01}
%	\caption{}
%	\label{fig:23601}
%\end{figure}

\section{Разработки функциональной схемы одноразрядного десятичного сумматора комбинационного типа}

\subsection{Разработка оптимальной схемы (с точки зрения критерия оптимальности) одноразрядного двоичного сумматора с учетом заданного базиса логических элементов}

\subsection{Разработка схемы коррекции}

\subsection{Разработка схемы одноразрядного десятичного сумматора}

\section{Разработка дополнительных схем для функционирования многоразрядного десятичного сумматора}

\subsection{Разработка преобразователя прямого кода в обратный для работы с отрицательными величинами}

\subsection{Разработка схемы, фиксирующей переполнение разрядной сетки}

\subsection{Разработка схемы для определения знака суммы}

\section{Разработки функциональной схемы многоразрядного десятичного сумматора}

\section{Разработка устройства управления для многоразрядного десятичного сумматора}

\subsection{Разработка входных и выходных регистров хранения числовой информации, участвующей в операции сложения}

\subsection{Разработка регистра признаков результата}

\subsection{Расчет временных параметров устройства управления}

\subsection{Разработка схемы для получения управляющих сигналов и схемы пуска выполнения операции сложения}

\section{Общая структура схемы многоразрядного десятичного сумматора комбинационного типа с устройством управления}

\section{Выводы по работе}

\end{document}